    \documentclass[a4paper,10pt,twocolumn,fleqn]{article}
    \usepackage{graphicx,url}
    \usepackage[brazil]{babel}   
    \usepackage[utf8]{inputenc} 
    \usepackage[T1]{fontenc} 
    \usepackage{mathtools}
    \usepackage{amsmath}
    \usepackage{amssymb}
    \usepackage{fontawesome}
    \usepackage{booktabs}
    \usepackage{adjustbox}
    \usepackage{subcaption}
    \captionsetup{compatibility=false}
    \usepackage{pgfplots}
    \usetikzlibrary{positioning, backgrounds}

    \usepackage{makecell}
    \sloppy
    %\hyphenpenalty=10000
    %\exhyphenpenalty=10000
    \usepackage{mathptmx} %times new roman
    \usepackage[labelsep=endash]{caption}
    \pgfplotsset{compat=1.15} 
    
    %\parindent 5mm
    %\parskip   5mm
    \setlength{\parindent}{5mm} %recuo de texto para cada paragráfo
    \usepackage{indentfirst} %identar 1 paragráfo
 
  %contar referências como capítulo
  \usepackage{etoolbox}
  \patchcmd{\thebibliography}{*}{}{}{}

    %setar margens do documentos
    \usepackage{geometry}
    \geometry{
        a4paper,
        left=20mm,
        top=20mm,
        right=20mm,
        bottom=25mm
    }
    %suprimir data do artigo
    \date{\vspace{-5ex}}
    \date{}
    
    \renewcommand{\thetable}{\Roman{table}}
  
  
    %tamanho do texto (2 colunas de 8cm + 1cm de espaço)  
    \textwidth         170mm   
    \columnsep         10mm %espaço entre colunas
    
    %ajuste dos títulos
    \usepackage{titlesec}
    \titleformat{\section}
    {\normalfont\Large\bfseries\itshape\filcenter}{\thesection\hspace{1mm}.}{1mm}{}
    \titleformat{\subsection}
    {\normalfont\Large\bfseries\itshape\filcenter}{\thesubsection\hspace{1mm}.}{1mm}{}
    \titleformat{\subsubsection}
    {\normalfont\Large\bfseries\itshape\filcenter}{\thesubsubsection\hspace{1mm}.}{1mm}{}
    
    \titlespacing\section{0pt}{12pt plus 4pt minus 2pt}{0pt plus 2pt minus 2pt}
    \titlespacing\subsection{0pt}{12pt plus 4pt minus 2pt}{0pt plus 2pt minus 2pt}
    \titlespacing\subsubsection{0pt}{12pt plus 4pt minus 2pt}{0pt plus 2pt minus 2pt}

    %header e footer do artigo
    \usepackage{fancyhdr}
    \pagestyle{fancy}
    \fancyhf{}
    \fancyhead[L]{IX Simpósio de Iniciação Científica, Didática e de Ações Sociais da FEI}
    \renewcommand{\headrulewidth}{1pt}


    \fancyfoot[R]{São Bernardo do Campo -- 2019}
    \renewcommand\footrule{\makebox[\textwidth]{\shadowfill}\\[-1.5\baselineskip]}
    \newcommand\shadowfill{%
        \leavevmode\leaders\hbox{\ooalign{%
                \vrule height 0pt depth 1pt width 1pt}%
        }\hskip\fill\kern0pt%
    }

    %ambiente resumo
    \def\resumo{\normalfont%
        \Large\bfseries\textit{Resumo}:\,\normalfont\normalsize%
    }

    %ambiente agradecimentos
    \def\agradecimentos{\normalfont%
        \centering\Large\bfseries\textit{Agradecimentos}\\\normalfont\normalsize\hspace{5mm}%
    }

    \author{NONONO$^{1}$ and NONONO$^{2}$% <-this % stops a space
        \thanks{*This work was not supported by any organization}% <-this % stops a space
        \thanks{$^{1}$NONONONO
            {\tt\small NONONONO@NONO.NONO}}%
        \thanks{$^{2}$NONONONO
            {\tt\small NONONONO@NONO.NONO}}%
    }    

    \begin{document} 
        
    %no título não remover o \vspace, é que ele que puxa para o topo da página
    \title{\vspace{-2em}MODELO PARA O RESUMO}
   
   
    %não apagar aqui
    \maketitle
    \thispagestyle{fancy}
    
    
    \begin{resumo}
    O resumo deve ser escrito de forma clara e objetiva descrevendo o que será apresentado no pôster. Este item deve apresentar as principais ideias de desenvolvimento do projeto e seus principais resultados (quando houver) não ultrapassando 10 linhas.
    \end{resumo}
    
    
         
    \section{Introdução}
    O trabalho deve ser escrito em português em até 2 páginas de tamanho A4, em duas colunas, usando espaçamento simples e letra "Times New Roman" de tamanho 10. As margens da folha estão definidas na Tabela I. O trabalho deve conter as seções: resumo; introdução; metodologia; resultados e conclusões (quando houver) além de referências bibliográficas. 
    O título do trabalho deve ser escrito em negrito, com letra maiúscula de tamanho 18 e estar centralizado. Não pode ter mais que 2 linhas. Não é necessário que o título do artigo seja o mesmo do projeto de iniciação.
    Títulos de seções e subseções devem ser numerados, separados por ponto, escritas em negrito e itálico de tamanho 12 e centralizados na coluna. Esta página já está no formato padrão podendo ser utilizada como modelo.

     
    Para utilizar citações, use o comando \verb|\cite{chave}| que dará o resultado \cite{inp:wang2006,inp:saska2006,inp:okada2011,inp:albad2017}. 
     
    O restante do texto explica como utilizar formatações (seção \ref{sec:formatacao}), ilustrações (seção \ref{sec:ilustracoes}).
    

    \section{Formatação da página}\label{sec:formatacao}
    Os nomes dos autores, das instituições e o endereço eletrônico para contato devem ser escritos em itálico, centralizado e com tamanho 10. A ordem numérica seqüencial para identificação deve ser colocada sobrescrito, do lado direito para o autor e do lado esquerdo para as instituições. Quando mais de um autor pertencer à mesma instituição, o número seqüencial correspondente deverá ser posicionado separando-se por vírgula.
    O corpo do trabalho deve ser escrito com caráter de tamanho 10, sem linhas em branco separando os parágrafos. Em cada novo parágrafo, a primeira linha deve ser deslocada em 0,5 cm, conforme modelo. As referências devem ser indicadas entre chaves [1] ao longo do texto e descrito no final do artigo citando: nomes dos autores (pode ser abreviado, no máximo três nomes e caso tenha mais nomes usar o et. al.), nome da revista, volume, ano da publicação e página ou nome do livro, editor e ano de publicação. No final do trabalho deve constar o número seqüencial do autor com a respectiva bolsa.
    
    Para trabalhar com equações, \verb|$\vec{x}_i=(x_{i1},x_{i2},\dots,x_{iD})$| dará
    o resultado $\vec{x}_i=(x_{i1},x_{i2},\dots,x_{iD})$. Ou use o ambiente \verb|equation|.
    
    
    Para citar equações, como aqui \eqref{eq:pso_posicao} use o comando \verb|eqref{chave_eq}|
    
    
    \begin{equation} \label{eq:pso_posicao}
    P_i(t+1)=P_i(t) +V_i(t+1)
    \end{equation}
    
    \subsection{Teste}
    para verificar se subtítulos estão ok!!
    

    \section{Ilustrações}\label{sec:ilustracoes}
    As figuras devem ser centralizadas e referenciadas seqüencialmente na parte inferior da mesma por Figura 1- seguidas do título. O tamanho da figura e das letras dentro das figuras devem estar legíveis e podem ser coloridas desde que apresentem boa qualidade de impressão como na figura \ref{fig:grafico}.
    
    
    \begin{figure}[ht]
        
        \begin{adjustbox}{width=.5\textwidth}
            
            \begin{tikzpicture}[framed]
            \centering
            \begin{axis}[
            ybar, axis on top,
            title={Increased Success},
            height=7cm, width=\textwidth,
            bar width=0.5cm,
            ymajorgrids, tick align=inside,
            major grid style={draw=white},
            enlarge y limits={value=0,upper},
            ymin=0, ymax=100,
            axis x line*=bottom,
            axis y line*=right,
            y axis line style={opacity=0},
            tickwidth=10pt,
            enlarge x limits=true, scale=1,
            legend style={
                at={(0.5,-0.2)},
                anchor=north,
                legend columns=-1,
                /tikz/every even column/.append style={column sep=0.5cm}
            },
            ylabel={Percentage (\%)},
            symbolic x coords={
                Match 1,Match 2,Match 3, Match 4, Match 5},
            xtick=data,
            nodes near coords={
                \pgfmathprintnumber[precision=1]{\pgfplotspointmeta}
            }
            ]
            \addplot [draw=none, fill=blue!30] coordinates {
                (Match 1,46.23)
                (Match 2,49.57) 
                (Match 3,51.12)
                (Match 4,57.23) 
                (Match 5,65.43) };
            
            
            \addplot [draw=none, fill=green!30] coordinates {
                (Match 1,53.77)
                (Match 2,50.43) 
                (Match 3,48.88)
                (Match 4,42.77) 
                (Match 5,34.57) };
            
            \addplot [draw=none, fill=red!50] coordinates {
                (Match 1,66.11)
                (Match 2,69.31) 
                (Match 3,61.25)
                (Match 4,67.11) 
                (Match 5,69.67) };
            
            
            \addplot [draw=none, fill=gray!70] coordinates {
                (Match 1,73.30)
                (Match 2,70.21) 
                (Match 3,78.19)
                (Match 4,72.77) 
                (Match 5,74.19) };            
            
            \legend{Original Pass, Original Block Pass, Increased Pass, Increased Block Pass}
            \end{axis}
            \end{tikzpicture}
        \end{adjustbox}
        
        \caption{Gráfico feito com latex.}
        \label{fig:grafico}
    \end{figure}
    
    As equações devem ser apresentadas no lado esquerdo e numeradas no lado direito entre parênteses.
    
    As tabelas devem ser referenciadas seqüencialmente por Tabela \ref{tab:tab_margem} - seguida do titulo na parte superior da mesma e centralizado. O texto da mesma deve ser centralizado, como na tabela \ref{tab:tab_margem}).
    
    \begin{table}[h]
        \caption{Formato da Página}
        \centering
        \label{tab:tab_margem}
        \begin{tabular}{@{}|l|c|@{}}
            \toprule
            \textbf{Margem} & \multicolumn{1}{l|}{\textbf{Tamanho (em cm)}} \\ \midrule
            Superior        & 2,0                                           \\ \midrule
            Inferior        & 2,5                                           \\ \midrule
            Esquerda        & 2,0                                           \\ \midrule
            Direita         & 2,0                                           \\ \midrule
            Coluna          & 1,0                                           \\ \bottomrule
        \end{tabular}
    \end{table}
    
    
    \section{Conclusões}\label{sec:conclusion}
    
    Destaque os principais resultados alcançados na pesquisa.
    
    Para referências, crie o arquivo .bib correspondente ao seu artigo (utilize o programa jabref para facilitar a tua vida) e usa os comandos:
    \begin{verbatim}
    \bibliographystyle{sicfei}
    
    \bibliography{sicfei}
    \end{verbatim}
    
    Para criar as referências.

    \bibliographystyle{sicfei}
    
    \bibliography{sicfei}

   
    \begin{agradecimentos}
    À instituição XXX pela realização das medidas ou
    empréstimo de equipamentos.
    1 Aluno de IC do Centro Universitário FEI (ou FAPESP,
    CNPq ou outra). Projeto com vigência de XX/18 a XX/19.   
    \end{agradecimentos} 
     
    \end{document}
